%%
% Please see https://bitbucket.org/rivanvx/beamer/wiki/Home for obtaining beamer.
%%
%\documentclass[11pt,aspectratio=169]{beamer}
%\documentclass[11pt,aspectratio=169,xcolor={dvipsnames}]{beamer}

\documentclass[11pt,aspectratio=169,xcolor={dvipsnames},hyperref={pdftex,pdfpagemode=UseNone,hidelinks,pdfdisplaydoctitle=true},usepdftitle=false]{beamer}
\usepackage{presentation}


\title{Lecture 2: CES - Cobb-Douglas economies}
\author{Juan Herre\~{n}o\\UCSD}

% Enter presentation title to populate PDF metadata:
\newcommand{\Cov}{\text{Cov}_\lambda}
\newcommand{\El}{\mathbb{E}_\lambda}
\newcommand{\mm}{\text{{\fontfamily{qzc}\selectfont m}}}

\begin{document}

\maketitle

\begin{frame}{Goals for Today}
\begin{itemize}
\item So far you have worked with economies where every agent is a price-taker
\item Highly unsatisfactory to study inflation.
\begin{itemize}
\item Inflation is an aggregated measure of individual price changes
\item Price changes obviously depend on the prices firms \textbf{choose}
\item Studying how firms choose prices in models where no one set prices is not promising
\end{itemize}
\item Today it is a ``framework'' lecture
\begin{itemize}
\item Firms will take input prices as given
\item Choose input combinations
\item And set prices subject to demand curves
\end{itemize}
\item We will do this in the most tractable framework we have: the Dixit-Stiglitz framework
\item The DS framework is widely used in macro/trade/IO/urban/labor/etc. You should know it.
\end{itemize}
\end{frame}


\begin{frame}{Digression}
\begin{itemize}
\item Pedagogically, economists teach perfect competition as a benchmark
\item ... and then imperfect competition as an extension
\item I personally think that is not the right approach
\item Imperfect competition is the \textbf{general} case
\item and perfect competition is a \textbf{limiting} scenario
\end{itemize}
\end{frame}


\begin{frame}{Model}
\begin{itemize}
\item Household problem
\begin{align*}
\max_{\{C_{it}\},N_t } \mathbb{E}_0 \sum_{t=0}^{\infty} \beta^t u(C_t,N_t)\\
\text{subject to:   } \int_0^1 P_{it} C_{it} + A_{t+1} \leq W_t N_t +  A_t(1+i)\\
C_t = \left(\int_0^1 C^{\frac{\theta-1}{\theta}}_{it} di \right)^{\frac{\theta}{\theta-1}}
\end{align*}
\item Note that $\theta \geq 1$ will give you the elasticity of substitution across varieties. Do not need a continuum
\item As $\theta \rightarrow \infty$, function is linear, goods are perfect substitutes. As $\theta \rightarrow 1$, you get Leontief, no substitutability.
\item $C_t$ is oftentimes called a Dixit-Stiglitz aggregator. 
\item In this case, each variety is atomistic. Changes in $C_{it}$ alone do not affect aggregates
\end{itemize}
\end{frame}


\begin{frame}{Result 1: Two stage budgeting}
\begin{itemize}
\item We could write a Lagrangean/Bellman to solve the problem directly
\item Turns out there is an easier way
\item If upper nest is separable (in this case $N$ is separable from any $C_{i}$ in $u$), and lower nest is homothetic (which DS aggregator is), then we can split the problem in two stages:
\begin{itemize}
\item Take $C$ as given, and solve the allocation of expenditure across varieties
\item Solve the outer nest using standard consumer theory
\end{itemize}
\end{itemize}
\end{frame}


\begin{frame}{First Stage: Cost minimization}
\begin{align*}
\min_{\{C_{it}\}} \int_0^1 P_{it} C_{it} di \text{   subject to:  } \left(\int_0^1 C^{\frac{\theta-1}{\theta}}_{it} di \right)^{\frac{\theta}{\theta-1}} \geq \bar{C}
\end{align*}
\begin{itemize}
\item Main advantage: Static problem with one constraint. Lagrange multiplier $\lambda$
\item FOC:
\begin{align*}
P_{it} = \lambda_t \left(\frac{C_{it}}{C_t}\right)^{-1/\theta}
\end{align*}
\item Take ratio of the FOC for two varieties $i$ and $j$
\begin{align*}
\frac{C_{it}}{C_{jt}} = \left(\frac{P_{it}}{P_{jt}}\right)^{-\theta}
\end{align*}
\item Making obvious that the elasticity of substitution between any two varieties is $\theta$.
\end{itemize}
\end{frame}


\begin{frame}{Price Indices}
\begin{itemize}
\item Summarize prices in the economy with a single index. Infinite choices. Two appealing ones:
\item The relative demand price index, which is the Lagrange multiplier $\lambda_t$.
\begin{align*}
\frac{C_{it}}{C_{t}} = \left(\frac{P_{it}}{\lambda_t}\right)^{-\theta}
\end{align*}
\item The ideal price index. Define $P_t$ as satisfying $$P_t C_t = \int_0^1 P_{it} C_{it} di $$
\item How do these compare?
\end{itemize}
\end{frame}

\begin{frame}{Result 2: The ideal price index = relative demand price index}
\begin{itemize}
\item Under CES these two appealing options are the same index
\item Evaluate the constraint $ \left(\int_0^1 C^{\frac{\theta-1}{\theta}}_{it} di \right)^{\frac{\theta}{\theta-1}} = C_t$, and plug the demand curve $\frac{C_{it}}{C_{t}} = \left(\frac{P_{it}}{\lambda_t}\right)^{-\theta}$
$$\lambda_t = \left(\int_0^1 P_{it}^{1-\theta} di \right)^{1/(1-\theta)}$$
\item Use the definition of the ideal price index $P_t C_t = \int_0^1 P_{it} C_{it} di $ and plug-in the demand curve
$$P_t = \lambda_t$$
\item This property is very specific to CES. Not necessarily true in other demand systems.
\end{itemize}
\end{frame}

\begin{frame}{Second stage: Utility maximization}
\begin{align*}
\max_{\{C_{it}\},N_t } \mathbb{E}_0 \sum_{t=0}^{\infty} \beta^t u(C_t,N_t)\\
\text{subject to:   } {\color{red}{P_tC_t}} + A_{t+1} \leq W_t N_t +  A_t(1+i)
\end{align*}
\begin{itemize}
\item Same problem you know
\item Used the definition $P_t C_t = \int_0^1 P_{it} C_{it} di$
\item In the background we have $$\frac{C_{it}}{C_{t}} = \left(\frac{P_{it}}{P}\right)^{-\theta}$$
\end{itemize}
Very tractable demand structure for an arbitrary number of goods.
\end{frame}


\begin{frame}{Monopolistic Competition}
\begin{itemize}
\item The assumption is that one, and only one, firm produces each variety $i$
\item So I can index a firm and a variety with the same index.
\item The firm is Neoclassical as in 210A
\begin{align*}
\max_{L_{it},K_{it},Y_{it},P_{it}} P_{it} Y_{it} - R_t K_{it} - w_t L_{it}
\end{align*}
\item But it has one additional constraint
\begin{align*}
Y_{it} \leq K_{it}^{\alpha}L_{it}^{1-\alpha}\\
Y_{it} = C_t \left(P_{it}/P_t\right)^{-\theta}
\end{align*}
\item Notice that monopolistic competition in a closed economy implies $Y_{it} = C_{it}$.
\item Can apply two-stage budgeting as well
\end{itemize}
\end{frame}

\begin{frame}{First Stage. Cost minimization}
\begin{align}
\min_{K_{it},L_{it}} R_t K_{it} + w_t L_{it} \text{    subject to:  }  K_{it}^{\alpha}L_{it}^{1-\alpha} \geq \bar{Y}  
\end{align}
\begin{itemize}
\item Lagrange multiplier $\psi_t$. FOC:
\begin{align*}
R_t K_{it} = \alpha Y_t \psi_t \\ w_t L_{it} = (1-\alpha) Y_t \psi_t
\end{align*}
\item Evaluate the constraint  $K_{it}^{\alpha}L_{it}^{1-\alpha} = Y_{it} $ at the optimum
$$\psi_t = \left(\frac{R_t}{\alpha}\right)^{\alpha} \left(\frac{w_t}{1-\alpha}\right)^{1-\alpha}$$
\item Total cost function $TC_{it} = R_t K_{it} + w_t L_{it}$ and evaluate at the optimum
$$TC_{it} = \psi_{it} Y_{it}$$
\item Making evident that
$$mc_{it} = \psi_t$$
\item Could have known that by simply remembering that the Lagrange multiplier is the effect on the cost function of a marginal increase in production. 
\end{itemize}
\end{frame}

\begin{frame}{Second Stage: Profit maximization}
\begin{itemize}
\item Profit maximization
\begin{align*}
\max_{Y_{it},P_{it}} (P_{it} - mc_{it})Y_{it} \\ \text{subject to:   } Y_{it} = C_t (P_{it}/P_t)^{-\theta}
\end{align*}
\item FOC and simplify:
\begin{align*}
P_{it} = \frac{\theta}{\theta-1} mc_{it} \equiv \mu \times mc_{it}
\end{align*}
\item Price is a constant markup over marginal cost. Profits are:
\begin{align*}
(\mu - 1)mc_{it} C_{it} \geq 0
\end{align*}
\item In the limit where $\theta \rightarrow \infty$, $\mu \rightarrow 1$ and $\text{Profits}_{it} \rightarrow 0$.
\end{itemize}
\end{frame}

\begin{frame}{Extensions}
\begin{itemize}
\item Framework amenable to extensions. Examples:
\begin{itemize}
\item Nested CES: Consumers choose among sectors. Sectors are composed of varieties. Two layer-CES.
\item Trade models: Consumers choose among origin of imports. Origin countries have different industries. Industries are composed by varieties. Three-layer CES.
\item Easy to extend to production functions with Decreasing Returns to scale.
\end{itemize}
\end{itemize}
\end{frame}

\end{document}






