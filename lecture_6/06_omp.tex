%%
% Please see https://bitbucket.org/rivanvx/beamer/wiki/Home for obtaining beamer.
%%
%\documentclass[11pt,aspectratio=169]{beamer}
%\documentclass[11pt,aspectratio=169,xcolor={dvipsnames}]{beamer}

\documentclass[11pt,aspectratio=169,xcolor={dvipsnames},hyperref={pdftex,pdfpagemode=UseNone,hidelinks,pdfdisplaydoctitle=true},usepdftitle=false]{beamer}
\usepackage{presentation}


\title{Lecture 6: Optimal Monetary Policy}
\author{Juan Herre\~{n}o\\UCSD}

% Enter presentation title to populate PDF metadata:
\newcommand{\Cov}{\text{Cov}_\lambda}
\newcommand{\El}{\mathbb{E}_\lambda}
\newcommand{\mm}{\text{{\fontfamily{qzc}\selectfont m}}}
\usepackage{cancel}
\begin{document}

\maketitle


\begin{frame}{Goals}
\begin{itemize}
\item How should monetary policy be conducted?
\begin{itemize}
\item Broad qualitative consensus among policymakers
\begin{itemize}
\item Central banks are in charge of maintaining low and stable inflation
\begin{itemize}
\item Explicit in the Federal Reserve Act of 1977
\item Explicit in the Maastricht treaty as the goal of the ECB
\item Written in the law and constitutions of countries around the world
\end{itemize}
\item Actual monetary policy assumes at least some degree of concern with stabilization of economic activity
\begin{itemize}
\item Explicit in the case of the US
\item Not explicit in the case of the ECB
\end{itemize}
\end{itemize}
\item Suggests the central bank wants to minimize a ``loss-function'' that depends on ``deviations'' of output and inflation from ``target''
\item Many questions
\begin{itemize}
\item What is the relative weight of inflation vs. output?
\item What is the right measure of economic activity to stabilize?
\item What is the right inflation index to stabilize?
\item Should the price level or the inflation rate be stabilized?
\end{itemize}
\end{itemize}
\end{itemize}
\end{frame}

\begin{frame}{Approach}
\begin{itemize}
\item In this lecture we will see what the NK model has to say about optimal monetary policy
\item The notion of optimality is to maximize the welfare of private agents
\item This \textit{utility-based} approach to policy design has a long tradition in public finance.
\begin{itemize}
\item Examples are the optimal capital tax rate, optimal unemployment insurance, ...
\end{itemize}
\item Notice our consumers do not care directly about prices. They care about leisure and consumption
\item But as taxes can create deadweight losses, inflation can too. We saw two potential sources in Lecture 5
\begin{itemize}
\item The aggregate markup can change, inducing too much or too little production
\item Price dispersion may change inducing allocative efficiency costs (misallocation). goods with the same marginal cost will have different prices
\end{itemize}
\end{itemize}
\end{frame}

\begin{frame}{Loss Function}
\begin{itemize}
\item The next few of slides provide a heuristic derivation of the Loss Function.
\item I show the math only to make clear the connection between economic objects
\item I will never ask you to know this math by heart in a final/qualifying exam
\item Not because it's too hard. Just because it's pointless for you to memorize it
\item I will just gloss over key steps in the lecture
\end{itemize}
\end{frame}

\begin{frame}
\begin{itemize}
\item Notation:
\begin{itemize}
\item $U_t$: $U(C_t,N_t)$.
\item $U_t^n$: $U_t$ evaluated at the flexible price allocation
\item $U$: $U_t$ in the steady state (no shocks whatsoever).
\item $U_c$ and $U_n$: partial derivatives of $U_t$ with respect to $C$, $N$ evaluated in the steady state. 
\item $U_{cc}$, $U_{nn}$: same thing for second derivatives.
\item Assume $U$ is separable in $C,N$. Therefore $U_{cn} = U_{nc} = 0$.
\end{itemize}
\end{itemize}
\end{frame}


\begin{frame}{Second-Order Approximation}
\begin{itemize}
\item I will do a second order approximation of the utility function around $U$
\begin{align*}
U_t \approx U + U_c C \frac{C_t - C}{C} + U_n N \frac{N_t - N}{N} + \frac{1}{2} U_{cc} C^2 \left(\frac{C_t - C}{C}\right)^2 + \frac{1}{2} U_{nn} N^2 \left(\frac{N_t - N}{N}\right)^2 
\end{align*}
\item For small deviations, log changes and percentage deviations are approximately equal
\item use our functional form assumption that $\gamma$ and $\varphi$ are $-C (U_{cc}/U_c)$ and $N (U_{nn}/U_n)$, respectively and impose market clearing $C = Y$.
\begin{align*}
U_t \approx U + U_c C \hat{y}_t + U_n N \hat{n}_t + \frac{1-\sigma}{2} U_{c}  \hat{y}^2_t + \frac{1+\varphi}{2} \hat{n}^2_t
\end{align*}
\item collect terms and use the results on price dispersion of Lecture 5. (Full details Gali Appendix 4.A)
\begin{align*}
\frac{U_t - U}{U_c C} \approx -\frac{1}{2} \left(\theta var_i ( \log P_{it}) + (\gamma + \varphi) \tilde{y}^2_t \right)
\end{align*}
\end{itemize}
\end{frame}


\begin{frame}{Price Dispersion and Inflation}
\begin{itemize}
\item Let me call $\Delta_t \equiv	var_i \log P_{it}$ and let me call $\bar{P}_t = \mathbb{E}_i \log P_{it}$
\item and note that due to Calvo.
$$\bar{P}_t - \bar{P}_{t-1} = \mathbb{E}_i (\log P_{it} - \bar{P}_{t-1}) = \cancelto{0}{\lambda \mathbb{E}_i (\log P_{it-1} - \bar{P}_{t-1})} + (1-\lambda)(\log P^*_t - \bar{P}_{t-1}) $$
\item Now $\Delta_t = var_i \log P_{it} = var_i (\log P_{it} - \bar{P}_{t-1})$ and use the definition of the variance
$$\Delta_t = \mathbb{E}_i \left[ (\log P_{it} - \bar{P}_{t-1})^2 \right] - (\cancelto{\bar{P}_t}{\mathbb{E}_i \log P_{it}} - \bar{P}_{t-1})^2$$
\item And the Calvo property
$$\Delta_t = \lambda \mathbb{E}_i \left[ (\log P_{it-1} - \bar{P}_{t-1})^2 \right] + (1-\lambda)(\log P_t^* - \bar{P}_{t-1})^2 -  (\bar{P}_t - \bar{P}_{t-1})^2$$
\item use the first result in the side
$$\Delta_t \approx \lambda \Delta_{t-1} + \frac{\lambda}{1-\lambda}  (\pi_t)^2$$
\item Since $\bar{P} \approx \log P_t$. See Woodford (2003) Appendix E.2 for precise mathematical statement.
\end{itemize}
\end{frame}

\begin{frame}{Welfare}
\begin{itemize}
\item An affine transformation of the objective of the household
$$\mathcal{W} = \mathbb{E}_0 \sum_{t=0}^{\infty} \beta^t \frac{U_t - U}{U_c C}$$
\item Use our second order approximation:
$$\mathcal{W} = -\frac{1}{2 }\mathbb{E}_0 \sum_{t=0}^{\infty} \beta^t \left(\theta var_i ( \log P_{it}) + (\gamma + \varphi) \tilde{y}^2_t \right)$$
\item Iterate $\Delta_t$ forward and compute its present value (Details Woodford Chapter 6 2.2).
$$\sum_{t=0}^{\infty} \beta^t \Delta_t \propto \frac{\lambda}{(1-\lambda)(1-\lambda\beta)} \sum_{t=0}^{\infty} \beta^t \pi^2_t$$
\item Finding
$$\mathcal{W} = -\frac{1}{2 }\mathbb{E}_0 \sum_{t=0}^{\infty} \beta^t \left(\frac{\theta}{\alpha}\pi^2_t + (\gamma + \varphi) \tilde{y}^2_t \right)$$
\end{itemize}
\end{frame}

\begin{frame}{Result 1: In the Calvo model price dispersion is very costly}
$$\mathcal{W} = -\frac{1}{2 }\mathbb{E}_0 \sum_{t=0}^{\infty} \beta^t \left(\frac{\theta}{\alpha}\pi^2_t + (\gamma + \varphi) \tilde{y}^2_t \right)$$
\begin{itemize}
\item Key Economics
\begin{itemize}
\item Households care about consumption and leisure.
\item They don't like consuming too little, or working too much vs. the efficient allocation
\item Output gap captures the differences
\item Price dispersion is a symptom of misallocation and it reduces welfare
\item Due to Calvo, inflation and price dispersion are tightly linked. This depends on Calvo!
\end{itemize}
\item Back of the envelope calculations
\begin{itemize}
\item Imagine utility is log in $C$ and linear in $N$. Then $\gamma + \varphi = 1$.
\item Typical values for  $\theta \in [4,7]$. Let's pick $\theta = 4$
\item If $\lambda = 0.9$ and $\beta = 0.995$, then $\alpha \approx 0.01$
\item So $\theta/\alpha \approx 330$
\item Inflation is waaaaaaaaay more costly than output gaps
\end{itemize}
\end{itemize}
\end{frame}


\begin{frame}{Optimal Policy}
$$\max_{\pi_t, \tilde{y}_t}\mathcal{W} = -\frac{1}{2 }\mathbb{E}_0 \sum_{t=0}^{\infty} \beta^t \left(\frac{\theta}{\alpha}\pi^2_t + (\gamma + \varphi) \tilde{y}^2_t \right)$$
\begin{itemize}
\item Subject to 
$$\pi_t = \beta \mathbb{E}_t \pi_{t+1} + \kappa \tilde{y}_t$$
$$\tilde{y}_t = \mathbb{E}_t \tilde{y}_{t+1} - \sigma(\hat{i}_t - \mathbb{E}_t \pi_{t+1} - \hat{r}^n_t)$$
\item We are looking for sequences for $\left\lbrace \hat{i}, \tilde{y}, \hat{\pi}\right\rbrace$ such that $\mathcal{W}$ is maximized
\end{itemize}
\end{frame}


\begin{frame}{Optimal Policy}
\begin{itemize}
\item It is useful to separate the problem in two
\begin{itemize}
\item Pick a sequence for  $\left\lbrace  \tilde{y}, \hat{\pi}\right\rbrace$ that maximizes $\mathcal{W}$ subject to the Phillips curve
$$\max_{\pi_t, \tilde{y}_t}\mathcal{W} = -\frac{1}{2 }\mathbb{E}_0 \sum_{t=0}^{\infty} \beta^t \left(\frac{\theta}{\alpha}\pi^2_t + (\gamma + \varphi) \tilde{y}^2_t \right)$$
Subject to 
$$\pi_t = \beta \mathbb{E}_t \pi_{t+1} + \kappa \tilde{y}_t$$
\item Find a sequence for  $\left\lbrace \hat{i}\right\rbrace$ that satisfies 
$$\tilde{y}_t = \mathbb{E}_t \tilde{y}_{t+1} - \sigma(\hat{i}_t - \mathbb{E}_t \pi_{t+1} - \hat{r}^n_t)$$
\end{itemize}
\item Solution
\begin{itemize}
\item $\hat{\pi}_t = 0$, $\hat{y}_t = \hat{y}^n_t$ $\forall t$ (so that $\tilde{y}_t = 0$) minimizes the objective and respects the Phillips curve
\item Pick $\hat{i}_t = \hat{r}^n_t$ $\forall t$
\end{itemize}
\end{itemize}
\end{frame}


\begin{frame}{Result 2: The Divine Coincidence}
\begin{itemize}
\item The central bank does not want to stabilize output, it wants to stabilize the output gap!
\item Divine Coincidence: The central bank can minimize output gap \textbf{and} inflation deviations simultaneously. No tradeoff.
\item Notice: Central bank does not want to induce monetary policy shocks. $\hat{i}_t = \hat{r}^n_t$ $\forall t$. Optimal policy 100\% systematic.
\item Notice: Optimal policy informationally very heavy. Central bank must now $r^n_t$ perfectly.
\item Shocks to the IS curve (demand shocks) are offset via movements of $i$
\end{itemize}
\end{frame}

\begin{frame}{The Divine Coincidence: Intuition}
\begin{itemize}
\item The flexible price allocation is optimal
\item Nominal rigidities are the only constraint to reach this allocation
\item If the constraint does not bind (happens with zero inflation), there is no distortion
\item So neutralizing inflation implies neutralization of output gap
\end{itemize}
\end{frame}


\begin{frame}{The Unrealistic Divine Coincidence}
\begin{itemize}
\item If you were to ask central bankers, they would not agree that stabilizing one objective necessarily stabilizes the other.
\item Rough response you would get. There are some shocks that increase inflation, and stabilizing inflation would push down output relative to the desired level
\item Simple way in this model to capture that intuition: if the flexible price equilibrium is not efficient
\end{itemize}
\end{frame}



\begin{frame}
\frametitle{Time-Varying Efficient Output Gap}
\begin{itemize}
	\item In our model so far, the outcome under flexible prices, $\hat{y}_t^{n}$, is also the efficient (first-best) outcome $\hat{y}_t^{eff}$.
	\item The central bank will face a trade-off and the divine coincidence will break once these are no longer the same.
	\item The welfare function will now penalize deviations of output from the efficient level of output $\hat{y}_t - \hat{y}_t^{eff}$. The Phillips Curve is:
		\begin{align*}
			\hat{\pi}_t&=\kappa (\hat{y}_t - \hat{y}_t^{n}) +\beta E_t \{\hat{\pi}_{t+1}\} \\
			&= \kappa (\hat{y}_t - \hat{y}_t^{eff}) +\beta E_t \{\hat{\pi}_{t+1}\} \underbrace{- \kappa (\hat{y}_t^{n} - \hat{y}_t^{eff})}_{\equiv u_t} \\
			&= \kappa (\hat{y}_t - \hat{y}_t^{eff}) +\beta E_t \{\hat{\pi}_{t+1}\} + u_t
		\end{align*}
%	\begin{itemize}
%		\item These are now no longer the same.
%	\end{itemize}
%	\item Let $\hat{x}_t = \hat{y}_t-\hat{y}_t^e$ be the gap between output and efficient 
%output that the planner wishes to stabilize.
%	\begin{itemize}
%		\item Then $\tilde{y}_t = (\hat{y}_t-\hat{y}_t^e)-(\hat{y}_t^e-\hat{y}_t^n)$
%		\item The Phillips Curve is then
%		\begin{align*}
%			\hat{\pi}_t&=\kappa \hat{x}_{t} +\beta E_t \{\hat{\pi}_{t+1}\} + u_t
%		\end{align*}
%		where $u_t=\kappa (\hat{y}_t^e-\hat{y}_t^n)$ is exogenous with respect to monetary policy.
%		\item Assume $u_t$ follows AR(1):
%		\begin{align*}
%			u_t=\rho_u u_{t-1}+\epsilon_t^u
%		\end{align*}
%	\end{itemize}
 	\item We call $u_t$ a ``cost-push shock.''
 	\begin{itemize}
 		\item Exogenous increase in marginal costs.
 	\end{itemize}
\end{itemize}
\end{frame}


\begin{frame}
\frametitle{Cost Push and the Labor Wedge}
\begin{itemize}
	\item What are cost push shocks?
	\begin{itemize}
		\item Anything that moves the labor wedge beyond sticky prices.
	\end{itemize}
%	\item Start with labor market frictions.
	\item Let $\mu_t^W$ be the log of a time-varying exogenous wage markup:
	\begin{align*}
		\hat{w}_t-\hat{p}_t=\hat{\mu}^W_t+\varphi \hat{n}_t+\gamma \hat{c}_t
	\end{align*}
%	\item Let $\mu_t^W$ be the log of a time-varying exogenous price markup.
	\item Then the Phillips curve becomes:
	\begin{align*}
		\hat{\pi}_t&=\kappa (\hat{y}_t - \hat{y}_t^{eff}) +\beta E_t \{\hat{\pi}_{t+1}\} +\alpha \hat{\mu}^W_t
	\end{align*}
	\item Intuition: 
	\begin{itemize}
		\item Higher mark-ups mean higher inflation and lower output.
		\item Central bank wants to offset this inefficient shock, but can only move output and inflation in the same direction.
%		\item But mark-ups are inefficient.
	\end{itemize}
\end{itemize}
\end{frame}


%\begin{frame}
%\frametitle{Cost Push and the Labor Wedge}
%\begin{itemize}
%	\item  Plug into
%	\begin{align*}
%		\hat{mc}_t&=\hat{w}_t-\hat{p}_t-\hat{a}_t \\
%		&=(\gamma+\varphi)\hat{y}_t - (1+\varphi)\hat{a}_t+\hat{\mu}^W_t
%	\end{align*}
%	\item Subtract efficient level of output:
%	\begin{align*}
%		(\gamma+\varphi)\hat{y}_t^e = (1+\varphi)\hat{a}_t
%	\end{align*}
%	to get
%	\begin{align*}
%		\hat{mc}_t&=(\gamma+\varphi)\hat{x}_t +\hat{\mu}^W_t
%	\end{align*}
%\end{itemize}
%\end{frame}
%
%\begin{frame}
%\frametitle{Cost Push and the Labor Wedge}
%\begin{itemize}
%	\item  Go back to the log-linearization of the optimal pricing condition but insert a variable markup $\hat{\mu}^P_t$ which represents ``markup shocks.''
%	\item Gali Appendix 5.2 shows
%	\begin{align*}
%		\hat{\pi}_t&=\lambda \hat{mc}_{t} +\beta E_t \{\hat{\pi}_{t+1}\} +\lambda \hat{\mu}^P_t
%	\end{align*}
%	\item Combining with $\hat{mc}_t$, we see
%	\begin{align*}
%		\hat{\pi}_t&=\kappa \hat{x}_{t} +\beta E_t \{\hat{\pi}_{t+1}\} +\lambda \hat{\mu}_t
%	\end{align*}
%	where $\hat{\mu}_t=\hat{\mu}_t^P+\hat{\mu}_t^W$ are shocks to the labor wedge through prices or wages.
%	\begin{itemize}
%		\item Important drivers of inflation according to DSGE models.
%	\end{itemize}
%\end{itemize}
%\end{frame}

\begin{frame}
\frametitle{The Planning Problem With A Tradeoff}
\begin{align*}
		\frac{1}{2}E_t\left\{\sum_{s=0}^{\infty}\beta^s\left[\vartheta (\hat{y}_{t+s}-\hat{y}_{t+s}^{eff})^2+\hat{\pi}_{t+s}^2\right]\right\}
	\end{align*}
	subject to
	\begin{align*}
		\hat{y}_t &=-\sigma E_t\{\hat{i}_t-\hat{\pi}_{t+1}\}+E_t\{\hat{y}_{t+1}\} \\
		\hat{\pi}_t&=\kappa(\hat{y}_{t} - \hat{y}_{t}^{eff}) +\beta E_t \{\hat{\pi}_{t+1}\} + u_t 
	\end{align*}	
\begin{itemize}
	\item First stage is optimizing objective function subject to NKPC.
	\begin{itemize}
		\item Cost push shock increases $\hat{\pi}_t$.
		\item To offset it, can push down output relative to efficient output $\hat{y}_{t} - \hat{y}_{t}^{eff}$.
		\item Thus there is now a tradeoff.
		\item Intuitively, monetary policy shifts aggregate demand but $u_t$ is a
shock to aggregate supply, so there is a tradeoff.
	\end{itemize}
\end{itemize}
\end{frame}


\begin{frame}
\frametitle{The Planning Problem: Rules vs. Discretion}
\begin{itemize}
	\item Standard quadratic loss function with linear constraints.
	\begin{itemize}
		\item But targets depend on expectations of future policy.
		\item To see this, iterate forward to get:
		\begin{align*}
			\hat{\pi}_t &=E_t\left\{\sum_{s=0}^{\infty}\beta^s[\kappa (\hat{y}_{t+s} - \hat{y}_{t+s}^{eff})+u_{t+s}]\right\} \\
			\hat{y}_t &=E_t\left\{\sum_{s=0}^{\infty}\left[-\sigma (\hat{i}_{t+s}-\hat{\pi}_{t+s+1})+g_{t+s}\right]\right\} 
		\end{align*}
	\end{itemize}
 	\item This raises issues of credibility and time consistency of policy.
 	\begin{itemize}
 		\item Central bank can influence outcomes today by ``promising'' outcomes tomorrow.
 		\item But are those promises credible? 
% 		\item Or would it want to renege on promises? How would ``optimal promises'' look and how would they help?
 		\item See Gali 5.3 for the solution to the commitment case.
 	\end{itemize}
 	\end{itemize}
\end{frame}

\begin{frame}
\frametitle{The Discretionary Problem}
\begin{itemize}
	\item For today, we assume that the central bank follows a \emph{discretionary optimal policy}.
	\begin{itemize}
		\item Cannot make credible commitments about future actions (hopefully have time to discuss why)
		\item So optimize \emph{taking  expectations of future actions as given}.
	\end{itemize}
	\item Solve
		\begin{align*}\min_{\hat{\pi}_t,\hat{y}_{t}}\frac{1}{2}[\vartheta (\hat{y}_{t} - \hat{y}_{t}^{eff})^2+\hat{\pi}_{t}^2]+F_t\text{ s.t. }\hat{\pi}_t=\kappa(\hat{y}_{t} - \hat{y}_{t}^{eff})+f_t
	\end{align*}
	where
	\begin{align*}
		F_t &=\frac{1}{2}E_t\left\{\sum_{s=1}^{\infty}\beta^s\left[\vartheta (\hat{y}_{t+s} - \hat{y}_{t+s}^{eff})^2+\hat{\pi}_{t+s}^2\right]\right\} \\
		f_t &= \beta\hat{\pi}_{t+1}+u_t
	\end{align*}
 	are functions of expectations of future actions.
 	\end{itemize}
\end{frame}


\begin{frame}
\frametitle{``Lean Against the Wind'' Policy}
\begin{align*}\min_{\hat{\pi}_t,\hat{y}_t}\frac{1}{2}[\vartheta (\hat{y}_{t} - \hat{y}_{t}^{eff})^2+\hat{\pi}_{t}^2]+F_t\text{ s.t. }\hat{\pi}_t=\kappa(\hat{y}_{t} - \hat{y}_{t}^{eff})+f_t
	\end{align*}
\begin{itemize}
	\item The First order condition is	\begin{align*}
		\hat{y}_{t} - \hat{y}_{t}^{eff} &=-\frac{\kappa}{\vartheta}\hat{\pi}_{t}
	\end{align*}
	\item  ``Lean Against The Wind'' Policy.
	\begin{itemize}
		\item In face of inflationary pressures from cost push shocks, \emph{drive output below its efficient level to dampen rise in inflation}.
		\item Extent to which it does so depends on:
		\begin{itemize}
			\item $\kappa$, which determines reduced inflation per unit of output loss.
			\item $\vartheta$, the relative weight placed on output loss.
		\end{itemize}
	\end{itemize}
	\item Flip from ``Old Keynesian'' logic where stabilizing output at cost of inflation.
\end{itemize}
\end{frame}


\begin{frame}
\frametitle{Inflation and Output Under Discretion}
\begin{itemize}
	\item Plug policy into Phillips curve:
	\begin{align*}
		\hat{\pi}_t &=\frac{\vartheta\beta}{\vartheta+\kappa^2}E_t\hat{\pi}_{t+1}+\frac{\vartheta}{\vartheta+\kappa^2}u_t
	\end{align*}
	\item  And iterate forward to get:
	\begin{align*}
		\hat{\pi}_t &=\frac{\vartheta}{\vartheta(1-\beta\rho_u)+\kappa^2}u_t
	\end{align*}
	\item Combine with optimality condition to get
	\begin{align*}
		\hat{y}_{t} - \hat{y}_{t}^{eff} &=-\frac{\kappa}{\vartheta(1-\beta\rho_u)+\kappa^2}u_t
	\end{align*}
	\item So central bank lets output gap and inflation fluctuate in proportion to current value of cost push shock.
	\begin{itemize}
		\item Intuition: Cost push increases inflation, central bank wants to smooth both so trades some inflation for output.
	\end{itemize}
\end{itemize}
\end{frame}

\begin{frame}
\frametitle{Interest Rate Under Discretion}
\begin{itemize}
	\item Plugging into dynamic IS:
	\begin{align*}
		\hat{y}_t &=-\sigma E_t\{\hat{i}_t-\hat{\pi}_{t+1}\}+E_t\{\hat{y}_{t+1}\} + g_t
	\end{align*}
	obtains
	\begin{align*}
		\hat{i}_t &=\hat{r}_{t+1}^{eff}+\phi_\pi \hat{\pi}_{t}+\frac{g_t}{\sigma}
	\end{align*}
	where
	\begin{align*}
		 \hat{r}_{t+1}^e = \frac{1}{\sigma}E_t\{\hat{y}_{t+1}^{eff} - \hat{y}_t^{eff}\},\qquad \phi_\pi=\rho_u+\frac{\kappa(1-\rho_u)}{\sigma[\vartheta(1-\beta\rho_u)+\kappa^2]}
	\end{align*}
	\item Central bank implements the optimal outcome with what looks like an interest rate (Taylor) rule.
\end{itemize}
\end{frame}


\end{document}






