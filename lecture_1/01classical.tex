%%
% Please see https://bitbucket.org/rivanvx/beamer/wiki/Home for obtaining beamer.
%%
%\documentclass[11pt,aspectratio=169]{beamer}
%\documentclass[11pt,aspectratio=169,xcolor={dvipsnames}]{beamer}

\documentclass[11pt,aspectratio=169,xcolor={dvipsnames},hyperref={pdftex,pdfpagemode=UseNone,hidelinks,pdfdisplaydoctitle=true},usepdftitle=false]{beamer}
\usepackage{presentation}


\title{Lecture 1: The Classical Dichotomy}
\author{Juan Herre\~{n}o\\UCSD}

% Enter presentation title to populate PDF metadata:
\newcommand{\Cov}{\text{Cov}_\lambda}
\newcommand{\El}{\mathbb{E}_\lambda}
\newcommand{\mm}{\text{{\fontfamily{qzc}\selectfont m}}}

\begin{document}

\maketitle

\begin{frame}{Questions for Today}
\begin{itemize}
\item What determines the return on nominal assets?
\item What is the classical dichotomy?
\item Why can the price level be indeterminate?
\item What determines inflation?
\item What is the Taylor Rule?
\end{itemize}
\end{frame}



\begin{frame}{The Chicken Economy}
You have worked with 	``real economies''. I will refer to them as ``chicken economies''
\begin{itemize}
\item The economy starts with an initial stock of $K_0$ chicken 
\item Firms mix existing chicken $K$ with labor $L$, in order to produce more chicken $Y$
\item Newly produced chicken can be broiled and eaten $C$, or placed next to alive chicken $I$. $Y = C + I$ must hold
\item The chicken fatality rate is $\delta$
\item The quantity of chicken available to produce chicken tomorrow is $K_{t+1} = K_t (1-\delta) + I_t$.
\item You must pay $r^b$ chickens for each chicken you borrow
\item You must pay $w$ chickens to your workers per hour of work
\item You can save in chicken, earning $r^l$ chickens next period
\end{itemize}
There is no notion of money in the chicken economy.
\end{frame}


\begin{frame}{Money}
\begin{itemize}
\item Some clarifications. I will not include a monetary base, quantity of money anywhere.
\item Instead we will work with a \textit{cashless} economy
\item Reasons:
\begin{itemize}
\item Theory: Theories where the central bank decides on money predict too volatile inflation.
\item Estimation: Money demand equations are very unstable. Seems to just be the wrong model.
\item Real-World relevance: Paper money is less and less important
\item Policy: Central banks do not decide policy by fixing a money supply. They issue reserves to target nominal interest rates, and are willing to exchange reserves for cash one-to-one to accomodate changes in money demand.
\item Who uses cash? Mostly, criminals.
\end{itemize}
\end{itemize}
\end{frame}

\begin{frame}{The Neutrality of Money}
\begin{itemize}
\item We will start working with \textit{nominal economies}. Some variables are determined in dollar terms, not in terms of chicken.
\item Is abstracting from money a fundamental theoretical drawback? Does money matter?
\begin{itemize}
\item The question is not: Does \textit{wealth} matter? The answer to that is obviously yes
\item Is the choice of unit of account, and the quantity of money, all else equal, of any relevance?
\end{itemize}
\item David Hume (1752) cautions:\\
\begin{quote}
\vspace{0.5cm}
If we consider any one kingdom by itself, it is evident, that the greater or less plenty
of money is of no consequence; since the prices of commodities are always proportioned to
the plenty of money, and a crown in Harry VII's time served the same purpose as a pound
does at present.
\end{quote}
\item Hume presents a more nuanced position later in his treatise
\end{itemize}
\end{frame}


\begin{frame}{The Model}
\begin{itemize}
\item Representative agent with preferences
\begin{align*}
\mathbb{E}_0 \left[ \sum_{t=0}^{\infty} \beta^t \log C_t \right]
\end{align*}
\item Can save in real bonds, and in nominal bonds:
\begin{align*}
P_t C_t + A_{t+1} + P_t K_{t+1} \leq P_t Y_t + A_t (1+i_{t-1})  + P_t K_t(1+r_{t-1})
\end{align*}
\item $Y_t$ an endowment. exogenous. stochastic. Markov.
\end{itemize}
Notice: two interest rates on two assets. Budget constraint is in dollars, not chicken.
\end{frame}

\begin{frame}{Real Bonds Euler Equation}
\begin{itemize}
\item Not going to go over the steps to derive FOCs. Life is too short.
\item For real bonds $K$:
\begin{align*}
\frac{1}{C_t} = \beta \mathbb{E}_t \left(\frac{1}{C_{t+1}}(1+r_t)\right)
\end{align*}
\item Nothing new, your old friend the Euler equation. if you decide to save in real bonds, same trade-offs as in 210A, 210B.
\item Log-linearizing the Euler equation around $C_t = C_{t+1} = C$ and $1+r = \beta^{-1}$:
\begin{align*}
\mathbb{E}_t (\hat{c}_{t+1} - \hat{c}_t) = \hat{r}_t
\end{align*}
The real interest rate $r$ pins down consumption growth
\end{itemize}
\end{frame}

\begin{frame}{The Fisher Equation}
\begin{itemize}
\item For nominal bonds:
\begin{align*}
1 = \beta \mathbb{E}_t \left(\frac{C_t}{C_{t+1}}\frac{P_t}{P_{t+1}}(1+i_t)\right)
\end{align*}
\item Euler equation for nominal bonds. Asset pricing equation $1 = \mathbb{E}(\text{SDF}_{t,t+1}R_{t+1})$
\item Combine with Euler for real bonds, to express as an asset pricing excess-return equation
\item For nominal bonds:
\begin{align*}
0 = \beta \mathbb{E}_t \left(\frac{C_t}{C_{t+1}}\left(1+r_t - \frac{P_t}{P_{t+1}}(1+i_t) \right)\right)
\end{align*}
\item Log-linearize around a s.s. with zero inflation and constant rates to make things more evident
\begin{align*}
\hat{i}_{t} = \hat{r}_t + \mathbb{E}_t \hat{\pi}_{t+1} \text{       : Fisher equation}
\end{align*}
\item where $\hat{\pi}_{t+1} = \hat{p}_{t+1} - \hat{p}_t$
\end{itemize}
\end{frame}


\begin{frame}{Market Clearing and Equilibrium}
\begin{itemize}
\item Goods market clear $$C_t = Y_t$$
\item There are no assets in net supply $$K_t = 0$$ $$A_t = 0$$
\end{itemize}
Equilibrium definition: An equilibrium is a sequence of real variables $\{r_{t+1},C_t,K_{t+1}\}$, and nominal variables $\{i_{t+1},A_{t+1},P_t\}$ starting at $t=0$, with initial conditions $B_0,K_0$, and an exogenous process for $Y_t$, such that
\begin{itemize}
\item The household behaves optimally: Euler equation and Fisher equation hold
\item Markets in goods and the two assets clear
\end{itemize}
Big problem: 5 equations, 6 unknowns.
\end{frame}


\begin{frame}{Result 1: Classical Dichotomy}
\begin{itemize}
\item The Euler equation for real bonds, the market clearing condition for goods, and for capital are a recursive system that pins down $Y, C,K$. No nominal variables enter the system.
\begin{align*}
K_{t+1} = 0\\
C_t = Y_t \\
(1+r_t)^{-1} =\beta \mathbb{E}_t \left(\frac{C_t}{C_{t+1}}\right) 
\end{align*}
\item Classical Dichotomy: real variables do not depend on any nominal variable.
\item A theoretical result, most evidence points against it. We will spend time breaking it.
\end{itemize}
\end{frame}


\begin{frame}{Result 2: Price Level Indeterminacy}
\begin{itemize}
\item Nominal holdings pinned down by $B_{t+1} = 0$
\item One equation left:
\begin{align*}
(1+i_t)^{-1} = \beta \mathbb{E}_t \left(\frac{C_t}{C_{t+1}} \frac{P_t}{P_{t+1}}\right)
\end{align*}
\item to pin down  $\{i_{t+1},P_t\}$.
\item David Hume seems to be right. Dollars are just a unit of account, and agents do not suffer from money illusion. Any sequence of $P$ as good as any other. Dollars or cents, who cares?
\item How to proceed?
\end{itemize}
\end{frame}


\begin{frame}{Pause}
Let's breathe for a second. Two results:
\begin{itemize}
\item Classical Dichotomy: Real equilibrium independent of nominal variables
\item Price level indeterminacy: Nothing tells you the price level. Any sequence of $P$ is ok.
\end{itemize}
These are two distinct concepts. To show it I will break the second while keeping the first.
\end{frame}

\begin{frame}{Fiscal Policy}
\begin{itemize}
\item Fiscal authority collects taxes and send transfers that add up to $T$. Purchases goods $G$, and receives dividends from the central bank $D$.
\item Balanced budget $$T_t + D_t = G_t$$
\item For simplicity assume $G$ is given, $D$ is decided by the central bank, so the government is forced to implement $T$ given by the budget constraint above.
\end{itemize}
\end{frame}


\begin{frame}{Central Bank}
\begin{itemize}
\item Central bank issues a liability. Called reserves. Stock of reserves $V$
\item The central bank promises a nominal return $i^v_t$
\item Budget constraint of the central bank
$$V_{t+1} = V_t(1+i^v_t) + P_t D_t$$
\item Supply of nominal assets must equal demand of nominal assets
$$A_t = V_t$$
\item Note the central bank chooses both $V$ and $i^v$. Central banks choose prices and quantities. Powerful.
\end{itemize}
\end{frame}

\begin{frame}{Real Equilibrium}
Real equilibrium same as before. Easy to check. Life is too short. Important: classical dichotomy still holds.
\end{frame}

\begin{frame}{Nominal Equilibrium}
\begin{itemize}
\item Nominal equilibrium is $\{P_t, i_t+1\}$ such that
$$i_t = i^v_t$$
$$(1+i_t)^{-1} = \beta \mathbb{E}_t \left(\frac{C_t}{C_{t+1}} \frac{P_t}{P_{t+1}}\right)$$
\item to pin down $P$ need to solve the following difference equation
$$r_t = i^v_t - \mathbb{E}_t \Delta p_{t+1}$$
\item Fisher equation extended for the central bank picking $i$.
\end{itemize}
\end{frame}

\begin{frame}{Result 3: Interest rate pegs do not work}
\begin{itemize}
\item Imagine the central bank picks an exogenous path for $i^v$. wlog $i^v = 0$.
\item Equilibrium then implies $$p_t = \mathbb{E}_t p_{t+1}  + r_t$$
\item Cannot solve by iterating forward. No initial or terminal condition for $p$
\item Indeterminacy of interest rate pegs: If the interest rate is exogenous, and I expect higher prices in the future, prices today jump. Any sequence of $p$ is ok.
\end{itemize}
\end{frame}

\begin{frame}{Result 4: Wicksellian rules}
\begin{itemize}
\item Instead assume $i^v_t = \phi (\hat{p}_t - \hat{p}_t^*)$, for some desired price $\hat{p}^*$
\item Fisher equation now $$(\phi + 1)\hat{p}_t = \mathbb{E}_t \hat{p}_{t+1} + r_t + \phi \hat{p}^*_t$$
\item Iterate forward assuming $\phi > 0$. $$\hat{p}_t = \mathbb{E}_t \sum_{s= 0}^{\infty} (1+\phi)^{-(s+1)}(r_{t+s} + \phi \hat{p}^*_{t+s})$$	 
\end{itemize}
Price level pinned down. If there are shocks to the expected path of real rates or desired targets, then the price level moves.
\end{frame}

\begin{frame}{Result 5: Taylor Rules and the Taylor Principle}
\begin{itemize}
\item Instead assume $i^v_t = \phi (\hat{\pi}_t - \hat{\pi}_t^*)$, for some desired inflation rate $\hat{\pi}^*$
\item Fisher equation now $$\phi \hat{\pi}_t = \mathbb{E}_t \hat{\pi}_{t+1} + r_{t+1} + \phi \hat{\pi}^*_t$$
\item Iterate forward assuming $\phi > 1$. $$\hat{\pi}_t = \mathbb{E}_t \sum_{s= 0}^{\infty} (\phi)^{-(s+1)}(r_{t+s+1} + \phi \hat{\pi}^*_{t+s})$$	 
\end{itemize}
Determining the inflation rate
\end{frame}


\begin{frame}{Taylor Rules and Taylor Principle}
$$ i^v_t = \phi (\hat{\pi}_t - \hat{\pi}_t^*), \text{   with:  } \phi >1$$
\begin{itemize}
\item That a Taylor rule satisfies $\phi>1$ is called the Taylor Principle
\item In words: If inflation increases, the central bank increases interest rates \textit{more than proportionally}
\item Was for years at the center of monetary policy debate:
\begin{itemize}
\item Is the Taylor Principle satisfied?
\item Is the Taylor Principle close to optimal monetary policy?
\item Are observed inflationary spirals violations of the Taylor Principle?
\end{itemize}
\item Note that both Wicksellian and Taylor rules required log-linearizations. They are a locally bounded equilibrium maintained under the threat that the central bank would blow up the economy. But maybe the economy can blow up? Why rule that out?
$$\uparrow \pi_t \rightarrow \uparrow i_t \rightarrow \uparrow \pi_{t+1} \rightarrow \uparrow i_{t+1} ...$$
\end{itemize}
\end{frame}



\begin{frame}{Taylor Rule in the US}
\begin{figure}
\centering
\includegraphics[width=0.85\textwidth]{figures/taylor_1}
\end{figure}
\end{frame}

\begin{frame}{Taylor Rule in the US}
\begin{figure}
\centering
\includegraphics[width=0.85\textwidth]{figures/taylor_2}
\end{figure}
\end{frame}

\begin{frame}{Questions for Today}
\begin{itemize}
\item What determines the return on nominal assets? {\color{red}{The Fisher Equation. An arbitrage condition}}
\item What is the classical dichotomy? {\color{red}{Monetary policy is neutral. Theoretical starting point}}
\item Why can the price level be indeterminate? {\color{red}{Because units are a veil. No money illusion}}
\item What determines inflation? {\color{red}{So far arbitrage and the ability of the government to issue liabilities and set its price}}
\item What is the Taylor Rule: {\color{red}{Hike interest rates aggresively when inflation occurs. Pins down inflation locally. Potential problems with inflationary explosions and the ZLB.}}
\end{itemize}
\end{frame}





\end{document}






