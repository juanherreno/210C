%%
% Please see https://bitbucket.org/rivanvx/beamer/wiki/Home for obtaining beamer.
%%
%\documentclass[11pt,aspectratio=169]{beamer}
%\documentclass[11pt,aspectratio=169,xcolor={dvipsnames}]{beamer}

\documentclass[11pt,aspectratio=169,xcolor={dvipsnames},hyperref={pdftex,pdfpagemode=UseNone,hidelinks,pdfdisplaydoctitle=true},usepdftitle=false]{beamer}
\usepackage{presentation}


\title{Lecture 4: The New Keynesian Model}
\author{Juan Herre\~{n}o\\UCSD}

% Enter presentation title to populate PDF metadata:
\newcommand{\Cov}{\text{Cov}_\lambda}
\newcommand{\El}{\mathbb{E}_\lambda}
\newcommand{\mm}{\text{{\fontfamily{qzc}\selectfont m}}}

\begin{document}

\maketitle

\begin{frame}{History of Economic Thought}
\begin{itemize}
\item First Keynesian models written down by Hicks, Modigliani, Samuelson, Solow, etc. Two inputs. Sluggish prices. Large MPCs.
\pause
\item ``Old Keynesian'' models heavily criticized in the 1970s by ``fresh-water'' economists
\begin{itemize}
\item Lack of role for expectations
\item Assumptions on the policy-invariance of aggregate elasticities in the data (Lucas' Critique).
\end{itemize}
\pause
\item Most of these fresh-water models:
\begin{itemize}
\item Representative Agent models
\item Introduction of Rational Expectations
\item RBC theory ignoring monetary policy at first. Classical dichotomy holds
\end{itemize}
\pause
\item Salt-water reply produced the ``New Keynesian'' model
\begin{itemize}
\item Spend the 1980s on micro-foundations of price rigidities
\item Simple representative agent rational expectation models but with frictions
\item Focus on: Inefficient fluctuations, so role for policy. Deviations from the classical dichotomy, so role for monetary policy
\end{itemize}
\end{itemize}
\end{frame}


\begin{frame}{Sticky prices}
\begin{itemize}
\item Intellectual figures: Ball, Blanchard, Calvo, Gali, Kiyotaki, Mankiw, D. Romer, Woodford
\pause
\item Maintained assumption that MPCs are small. Relaxed in new generation ``HANK'' models
\pause
\item Key innovation. Frictions to reset prices.
\pause
\item Not obvious how to do it. Alternatives.
\pause
\begin{itemize}
\item An aggregate price level is fixed. Not desirable. Assuming the result.
\pause
\item Are product prices or input prices subject to frictions? Sticky prices or sticky wages?
\pause
\item Firms face ``costs'' to reset their prices. Quadratic cost (Rotemberg), fixed cost (menu cost), random arrival (Calvo)
\pause
\item Firms need to pay to acquire information. Random arrival (Mankiw Reis), Information technological frictions (Angeletos La'O), convex cost of paying attention (Sims)
\end{itemize}
\pause
\item We will discuss many of them. But we will start with the Calvo model. Why? It is simple and honestly beautiful. Theoretical starting point.
\end{itemize}
\end{frame}

\begin{frame}{Calvo model}
\begin{itemize}
\item First proposed by Guillermo Calvo. Perhaps the most important Latin American economist in history. Worth reading about him. \pause
\item First are monopolistic competitors facing CES demand. You are experts.
\pause
\item Simplest production structure. Production linear in labor.
\pause
\item Price rigidity
\begin{itemize}
\item With probability $1-\lambda$ firms can adjust their prices. With probability $\lambda$ must keep them fixed
\pause
\item i.i.d across firms, and constant through time
\pause
\item $(1-\lambda)$ firms adjust today. $\lambda(1-\lambda)$ adjusted last period for the last time. $\lambda(1-\lambda)^2$ two periods ago, and so on.
\pause
\item Not meant to be realistic. We will contrast with the data soon. Will offer theoretical insights.
\pause
\item You often hear about the ``Calvo fairy'', the exogenous occurrence of a chance to reset prices
\end{itemize}
\end{itemize}
\end{frame}

\begin{frame}{Intuition}
Imagine this problem

\begin{align}
& \max_{P_{it},Y_{it}, L_{it}} P_{it} Y_{it} - W_t L_{it}\\
& \text{subject to:   } Y_{it} = C_t (P_{it}/P_t)^{-\theta} \text{  ,  } Y_{it} = A_t L_{it}
\end{align}
\begin{itemize}
\pause
\item Super simple. We solved it already in lecture 2.
\pause
\item Optimal pricing rule. Markup over marginal cost.
\begin{align*}
P_{it} = \frac{\theta}{\theta-1} \frac{W_t}{A_t}
\end{align*}
\pause
\item In this model firms want to change their prices because their marginal cost changes.
\pause
\item This pricing block static. Why worry about tomorrow when I can change my price every period, my inputs are spot, and demand curves static.
\end{itemize}
\end{frame}


\begin{frame}{Dynamic pricing}
\begin{itemize}
\item Now imagine instead that you have to pick a price $P_{it}^*$, that will apply today, and will still be in place tomorrow with probability $\lambda$, the day after tomorrow with probability $\lambda^2$, and so on.
\pause
\item Gives rise to a dynamic pricing problem. My price today will impact my profits in the future
\begin{align*}
&\max_{P^*_{it},Y_{i,t+k|t}, L_{i,t+k|t}} \mathbb{E}_t \sum_{k=0}^{\infty} \Lambda_{t,t+k} \lambda^k \left(P^*_{it} Y_{i,t+k|t} - W_{t+k} L_{i,t+k|k}\right) \frac{1}{P_{t+k}}\\
& \text{subject to:  } Y_{i,t+k|t} = C_{t+k} (P_{i,t+k|t}/P_{t+k})^{-\theta} \text{  ,  } Y_{i,t+k|t} = A_{t+k} L_{i,t+k|t}
\end{align*}
\pause
\item Firms take decisions under uncertainty (210A), discounting future profits using the SDF of the marginal investor $\Lambda$ (210B), by choosing a sticky price (lecture 4), subject to a sequence of CES demand curves (lecture 2).
\pause
\item Notice: Must satisfy demand. No such thing as ``close shop'' when the price is not good.
\pause
\item Notice: Is it ok to maximize this objective as opposed to the Value of the firm (that includes all the contingencies where it can adjust prices?)
\pause
\end{itemize}
\end{frame}


\begin{frame}{Optimality}
\begin{itemize}
\item Plug the constraints in the objective and take FOC with respect to $P_{it}^*$. Will not do it here. Life is too short.
\begin{align*}
P^*_{it} = \mathcal{M} \frac{\mathbb{E}_t \sum_{k=0}^{\infty} \lambda^k \Lambda_{t,t+k} \frac{W_{t+k}}{A_{t+k}} C_{t+k} P_{t+k}^{\theta-1}}{\mathbb{E}_t \sum_{k=0}^{\infty} \lambda^k \Lambda_{t,t+k} C_{t+k} P_{t+k}^{\theta-1}}
\end{align*}
\pause
\item where $\mathcal{M} = \frac{\theta}{\theta-1}$ is the gross markup under flexible prices, and $W/A$ is the nominal marginal cost
\item In words: The firm chooses $P^*$ to minimize the weighted distance of its profits to those of the flexible price eq., using as weights the probability that the price is active in period $k$, and the expected valuation of dividends by its owner in that period.
\pause
\item Notice: $P^*_{it}$ is the same $\forall i$. Result of assumptions on the nature of shocks and competition.
\end{itemize}
\end{frame}

\begin{frame}{Aggregation}
\begin{itemize}
\item $(1-\lambda)$ share of firms pick the same $P^*_{it} = P^*_t$.
\pause
\item $\lambda$ share of firms keep their price constant
\pause
\item Since resetters are picked at random, the average price of firms with fixed prices is $P_{t-1}$.
\pause
\item Important economic assumption. No selection in price changes!
\pause
\item Easy to show
$$P_t^{1-\theta} = (1-\lambda) (P^*_t)^{1-\theta} + \lambda P^{1-\theta}_{t-1}$$
\end{itemize}
\end{frame}

\begin{frame}{Household}
\begin{itemize}
\item Very easy. We pretty much did it in lecture 1 and 2
\item Representative agent with preferences
\begin{align*}
\mathbb{E}_0 \left[ \sum_{t=0}^{\infty} \beta^t \frac{C^{1-\gamma}_t}{1-\gamma} - \chi \frac{N^{1+\varphi}_t}{1+\varphi} \right]
\end{align*}
\pause
\item Can save in nominal bonds:
\begin{align*}
P_t C_t + B_{t+1}  \leq W_t N_t + B_t (1+i_{t-1}) 
\end{align*}
\pause
\item And a CES preference bundle and price index in the background
\begin{align*}
C_t = \left(\int_0^1 C^{\frac{\theta}{\theta-1}}_{it} di \right)^{\frac{\theta}{\theta-1}} \text{     ,    } P_t = \left(\int_0^1 P^{1-\theta}_{it} di \right)^{\frac{1}{1-\theta}} 
\end{align*}
\end{itemize}
\end{frame}

\begin{frame}{Optimality Conditions}
\begin{itemize}
\item Labor supply
\begin{align*}
\frac{W_t}{P_t} = \chi N^{\varphi}_t C_t^{\gamma}
\end{align*}
\pause
\item Euler equation for nominal bonds
\begin{align*}
1 = \mathbb{E}_t \left( \left(\frac{C_{t+1}}{C_t}\right)^{-\gamma} \frac{P_t}{P_{t+1}}(1+i_t)\right)
\pause
\end{align*}
\item And CES demand curves in the background
\begin{align*}
C_{it} = C_t \left(\frac{P_{it}}{P_t}\right)^{-\theta}
\end{align*}
\end{itemize}
\end{frame}


\begin{frame}{Monetary Policy}
\begin{itemize}
\item Very similar to the central bank we covered in lecture 1.
\pause
\item Sets interest rates. Follows a Taylor Rule.
\pause
\item  Random disturbances to nominal rates, like the monetary policy shocks in Lecture 3.
\begin{align*}
(1+i_t) = \beta^{-1} \left(\frac{P_t}{P_{t-1}}\right)^{\phi_{\pi}} e^{v_t}
\end{align*}
\pause
\item More realistic versions include responses to output
\pause
\item Not a statement on policy optimality. We will do that soon. Taylor rule more of an approximation to actual behavior
\end{itemize}
\end{frame}

\begin{frame}{Market clearing}
\begin{itemize}
\item Labor markets clear. $N_t = \int_0^1 L_{it} di$
\pause
\item Good markets clear $C_{it} = Y_{it}$
\pause
\item Bond markets clear $B_t = 0$
\pause
\item We will \textbf{define} aggregate output as $Y_t = \left(\int_0^1 Y^{\frac{\theta}{\theta-1}}_{it} di \right)^{\frac{\theta}{\theta-1}}$
\item Why? It achieves $Y_t = C_t$.
\end{itemize}
\end{frame}

\begin{frame}{Equilibrium Definition}
An equilibrium is an allocation $\left\lbrace C_{i,t+s} C_{t+s}, N_{t+s},Y_{t+s}\right\rbrace_{s=0}^{\infty}$, and a set of prices $\left\lbrace i_{t+s}, W_{t+s},P_{i,t+s} P_{t+s} \right\rbrace_{s=0}^{\infty}$, along with exogenous processes $\left\lbrace v_{t+s},  A_{t+s} \right\rbrace_{s=0}^{\infty}$ such that
\begin{itemize}
\item Households optimize: Labor leisure, euler equation, and demand curves are satisfied taking prices as given.
\item Firms optimize: Firm-prices are set optimally, price aggregation holds
\item Central bank sets policy according to the Taylor rule
\item Labor, bonds, and goods markets clear.
\end{itemize}
\end{frame}

\begin{frame}{Log-linearize}
\begin{itemize}
\item The model so far is difficult to analyze, too many non-linear equations
\pause
\item We will log-linearize the model. Paula taught you how. Need to choose a linearization point. We will choose the equilibrium with flexible prices and zero inflation
\pause
\item I will skip most of the log-linearization steps, as they are very standard. Check Gali's book if you are unsure how to do it.
\end{itemize}
\end{frame}



\begin{frame}
\frametitle{Log Linearization: Inflation
}
\begin{itemize}
	\item Price Index:
	\begin{align*}
		P_t&=\left[\lambda P_{t-1}^{1-\theta} + (1-\lambda) P_{t}^{*1-\theta}\right]^{\frac{1}{1-\theta}} 
	\end{align*}
	\pause
	\item The price index can be log-linearized to get
	\begin{align*}
		\hat{p}_t=\lambda \hat{p}_{t-1}+(1-\lambda)\hat{p}_t^*
	\end{align*}
	\pause
	\begin{itemize}
		\item Equivalently written in terms of inflation:
		\begin{align*}
		\hat{\pi}_t=(1-\lambda)(\hat{p}_t^*-\hat{p}_{t-1})
	\end{align*}
	\item Inflation is positive when newly set prices are higher than old prices.
	\end{itemize}
\end{itemize}
\end{frame}


\begin{frame}
\frametitle{Log Linearization: Reset Prices
}
\begin{itemize}
	\item The reset price can be log-linearized as:
	\begin{align*}
		\hat{p}_t^*=(1-\beta\lambda)E_t\left\{\sum_{s=0}^{\infty}(\beta\lambda)^s\left(\hat{p}_{t+s}+\hat{mc}_{t+s}\right)\right\} 
	\end{align*}
	\item Intuition: price deviations are equal to the expected future deviations of desired prices (real marginal costs + price index)
	\pause
	\item We can write this recursively as:
	\begin{align*}
		\hat{p}_t^*=(1-\beta\lambda)(\hat{p}_{t}+\hat{mc}_{t})+\beta\lambda E_t\{\hat{p}_{t+1}^*\}
	\end{align*}
\end{itemize}
\end{frame}


\begin{frame}
\frametitle{Log Linearization: Phillips Curve
}
\begin{itemize}
	\item Subtract $\hat{p}_{t-1}$:
	\begin{align*}
		(\hat{p}_t^*-\hat{p}_{t-1})=(1-\beta\lambda)\hat{mc}_{t}+\hat{\pi}_t+\beta\lambda E_t\{\hat{p}_{t+1}^*-\hat{p}_{t}\}
	\end{align*}
	\pause
	\item Plug into $\hat{\pi}_t=(1-\lambda)(\hat{p}_t^*-\hat{p}_{t-1})$ to get an expectations-augmented Phillips curve:
	\begin{align*}
		\hat{\pi}_t=\alpha\hat{mc}_{t}+\beta E_t\{\hat{\pi}_{t+1}\},\;\text{ where }\;\alpha=\frac{(1-\lambda)(1-\beta\lambda)}{\lambda}
	\end{align*}
	\pause
	\item Inflation is equal to expected future inflation plus the deviation of marginal cost from its steady state level.
	\begin{itemize}
		\item Expected inflation: Forward looking price setters choose higher prices now if inflation is expected to be high, as nominal marginal costs will rise.
	\end{itemize}
\end{itemize}
\end{frame}


\begin{frame}
\frametitle{Log Linearization: Phillips Curve
}
\begin{itemize}
	\item Inflation is equal to expected future inflation plus the deviation of marginal cost from its steady state level.
	\begin{itemize}
		\item Two ways to think about marginal cost deviation:
	\end{itemize}
	\begin{itemize}
		\item Set higher prices to cover higher marginal cost.
		\item When marginal costs are above desired level, markups are below desired level. Inflation as firms hike markup back to desired level. (In fact, $\hat{mc}_t = -\hat{\mu}_t$).
	\end{itemize}
	\item Iterating forward,
	\begin{align*}
		\hat{\pi}_t=\alpha E_t\left\{\sum_{s=0}^{\infty}\beta^s\hat{mc}_{t+s}\right\}
	\end{align*}
	\begin{itemize}
		\item Inflation is the PDV of future marginal cost / markup deviations from steady state.
	\end{itemize}
\end{itemize}
\end{frame}


\begin{frame}
\frametitle{Log Linearization: Real Marginal Costs
}
\begin{align*}
	\hat{mc}_t=\hat{w}_t-\hat{p}_t-\hat{a}_t
\end{align*}
\begin{itemize}
	\item Combine labor-leisure, production function, and $\hat{c}_t=\hat{y}_t$:
	\begin{align*}
		\hat{w}_t-\hat{p}_t=(\gamma+\varphi)\hat{y}_t-\varphi\hat{a}_t
\end{align*}
	\item Consequently,
	\begin{align*}
	\hat{mc}_t=(\gamma+\varphi)\hat{y}_t-(1+\varphi)\hat{a}_t
\end{align*}
\item Compare to flexible price case:
\begin{align*}
	1 = \frac{P_t(i)}{P_t} = \mu\frac{W_t}{P_t}\frac{1}{A_t} 
\end{align*}
so
\begin{align*}
	\hat{mc}_t^{flex} = 0,\qquad (\gamma+\varphi)\hat{y}_t^{flex}&=(1+\varphi)\hat{a}_t \\
\end{align*}
where $\hat{y}_t^{flex}$ is called the \emph{natural level of output}.
\end{itemize}
\end{frame}


%\begin{frame}
%\frametitle{Nonlinear Equations: Phillips Curve
%}
%\begin{itemize}
%	\item Putting everything together, the NK model features an \emph{expectations-augmented Phillips curve}.
%	\begin{itemize}
%		\item Because price setting is forward looking, Phillips curve is going to be \emph{expectation-augmented}.
%		\item Output-inflation relationship relates inflation to deviations from flex price output (called the \emph{natural level of output}).
%		\begin{itemize}
%			\item At natural level of output with no expected inflation, resetters choose price equal to flexible price, resulting in no inflation.
%			\item If output deviates, that is if the output gap $\tilde{Y}_t = (Y_t-Y_t^{flex})/Y_t^{flex}$ is nonzero, the extra output will bid up nominal wages and move marginal costs, causing inflation in response to deviations of output from its flexible level.
%		\end{itemize}
%	\end{itemize}
%	\item NK model a response to Lucas Critique.
%	\begin{itemize}
%		\item Unlike ``old'' Keynesian models, only unexpected inflation moves output.
%		\item Central bank cannot regularly exploit output-inflation trade-off.
%	\end{itemize}
%\end{itemize}
%\end{frame}


%\begin{frame}
%\frametitle{Log Linearization: Flexible Price Equilibrium
%}
%\begin{align*}
%	\frac{W_t^{flex}}{P_t^{flex}}&=\frac{\chi (N_t^{flex})^\varphi}{(C_t^{flex})^{-\gamma}} \\
%%	1&=\beta E_t\left\{Q_t \frac{P_t}{P_{t+1}} \frac{C_{t+1}^{-\gamma}}{C_{t}^{-\gamma}}\right\}=E_{t}\{\Lambda_{t,t+1}R_{t+1}\} \\
%%	P_t&=\left[\theta P_{t-1}^{1-\epsilon} + (1-\theta) P_{t}^{*1-\epsilon}\right]^{\frac{1}{1-\epsilon}} \\
%	\frac{W_t^{flex}}{P_t^{flex}} &=  \frac{A_t}{1+\mu} \\
%	Y_t^{flex}&=C_t^{flex} \\
%Y_t^{flex}&=A_t^{flex}N_t^{flex}
%%\\
%%	Q_t &= \beta^{-1}\left(\frac{P_t}{P_{t-1}}\right)^{\phi_{\pi}}\left(\frac{Y_t}{\bar{Y}}\right)^{\phi_{y}}e^{v_t}
%\end{align*}
%\begin{itemize}
%	\item Combine to get:
%	\begin{align*}
%		A_t^{1+\varphi}&=\chi(1+\mu)(Y_t^{flex})^{\gamma+\varphi} \\
%	(\gamma+\varphi)\hat{y}_t^{flex}&=(1+\varphi)\hat{a}_t
%\end{align*}
%\end{itemize}
%\end{frame}
%
\begin{frame}
\frametitle{Real Marginal Costs in Terms of Output Gap
}
\begin{itemize}
	\item Combine:
	\begin{align*}
		\hat{mc}_t&=(\gamma+\varphi)\hat{y}_t-(1+\varphi)\hat{a}_t \\
		(\gamma+\varphi)\hat{y}_t^{flex}&=(1+\varphi)\hat{a}_t \\
\end{align*}
	to write real marginal costs in terms of output gap $\tilde{y}_t$:
	\begin{align*}
		\hat{mc}_t&=(\gamma+\varphi)(\hat{y}_t-\hat{y}_t^{flex})
	\end{align*}
	\item Real marginal costs go up (and markups go down) when the output gap is high.
	\begin{itemize}
		\item To produce more than under flex prices, markup must be lower.
		\item Marginal costs high because need to hire more workers,
bidding up real wage.
		\item Stronger when IES and labor supply elasticity are low.
		\item In Gali textbook also stronger with DRS.
	\end{itemize}
\end{itemize}
\end{frame}

%
%\begin{frame}
%\frametitle{The New Keynesian Phillips Curve
%}
%\begin{itemize}
%	\item Plug back into the Phillips curve $\hat{\pi}_t=\lambda\hat{mc}_{t}+\beta E_t\{\hat{\pi}_{t+1}\}$
%	\begin{align*}
%		\hat{\pi}_t=\kappa\tilde{y}_t+\beta E_t\{\hat{\pi}_{t+1}\},\;\text{ where }\;\kappa=\lambda(\gamma+\varphi)
%\end{align*}
%	\begin{itemize}
%		\item This is the \emph{New Keynesian Philips Curve}: an expectations augmented Phillips curve written in terms of the output gap.
%	\end{itemize}
%	\item Solving forward,
%	\begin{align*}
%		\hat{\pi}_t=\kappa E_t\left\{\sum_{s=0}^{\infty}\beta^s\tilde{y}_{t+s}\right\}
%	\end{align*}
%	\begin{itemize}
%		\item Inflation is an increasing function of future output gaps.
%		\item Output gap high $\Rightarrow$ marginal cost high and markups low $\Rightarrow$ raise markups.
%	\end{itemize}
%\end{itemize}
%\end{frame}


\begin{frame}
\frametitle{The New Keynesian Phillips Curve
}
\begin{itemize}
	\item Plug back into the Phillips curve $\hat{\pi}_t=\alpha\hat{mc}_{t}+\beta E_t\{\hat{\pi}_{t+1}\}$
	\begin{align*}
		\hat{\pi}_t=\kappa(\hat{y}_t-\hat{y}_t^{flex})+\beta E_t\{\hat{\pi}_{t+1}\},\;\text{ where }\;\kappa=\alpha(\gamma+\varphi)
\end{align*}
	\begin{itemize}
		\item This is the \emph{New Keynesian Philips Curve}: an expectations augmented Phillips curve written in terms of the output gap.
		\item It is the aggregate supply curve of the model
	\end{itemize}
	\item Solving forward,
	\begin{align*}
		\hat{\pi}_t=\kappa E_t\left\{\sum_{s=0}^{\infty}\beta^s(\hat{y}_{t+s}-\hat{y}_{t+s}^{flex})\right\}
	\end{align*}
	\begin{itemize}
		\item Inflation is an increasing function of future output gaps.
		\item Output gap high $\Rightarrow$ marginal cost high and markups low $\Rightarrow$ raise markups.
	\end{itemize}
\end{itemize}
\end{frame}


%%%%%%%%%%%%%%%%%%%%%%%%%%%%%%%%%%%%%%%%%%%%%%%%%%
%\section{AD Block}
%%%%%%%%%%%%%%%%%%%%%%%%%%%%%%%%%%%%%%%%%%%%%%%%%%

\begin{frame}
\frametitle{Log Linearization: The Aggregate Demand Block
}
\begin{itemize}
	\item Log-linearize Euler around zero-inflation:
	\begin{align*}
		\hat{c}_t=-\frac{1}{\gamma}\left(\hat{i}_t-E_t\{\hat{\pi}_{t+1}\}\right)+E_t\{\hat{c}_{t+1}\}
	\end{align*}
	\begin{itemize}
		\item Steady state nominal interest rate is $i_t = \rho = -\log \beta$.
	\end{itemize}
	\item Combine with market clearing and use $\sigma=1/\gamma$:
	\begin{align*}
		\hat{y}_t=-\sigma\left(\hat{i}_t-E_t\{\hat{\pi}_{t+1}\}\right)+E_t\{\hat{y}_{t+1}\}
	\end{align*}
	\item This is the \emph{dynamic IS curve}. It relates output to future expectations of output and the real interest rate.
\end{itemize}
\end{frame}


%\begin{frame}
%\frametitle{The Natural Rate of Interest
%}
%\begin{itemize}
%	\item Define the natural rate of interest $\hat{r}_{t+1}^n$ as the real interest rate that would prevail when output is equal to its flexible level:
%	\begin{align*}
%		\hat{y}_t^{flex}=-\sigma \hat{r}_{t+1}^n+E_t\{\hat{y}_{t+1}^{flex}\}
%	\end{align*}
%	\item Recall $\hat{y}_t^{flex}=\left(\frac{1+\varphi}{\gamma+\varphi}\right)\hat{a}_t$ so:
%	\begin{align*}
%		\hat{r}_{t+1}^n&=\gamma \frac{1+\varphi}{\gamma+\varphi}E_t\{\hat{a}_{t+1}-\hat{a}_t\} \\
%		&=\gamma \frac{1+\varphi}{\gamma+\varphi}E_t\{\Delta \hat{a}_{t+1}\} 
%	\end{align*}
%	\begin{itemize}
%		\item If $a_t$ follows an AR(1) and grows today, it will be expected to decline between today and tomorrow due to mean reversion.
%		\item So positive tech shock causes real interest rate to fall by standard RBC logic.
%	\end{itemize}
%\end{itemize}
%\end{frame}
%
%
\begin{frame}
\frametitle{Dynamic IS
}
%\begin{align*}
%		\hat{y}_t=-\sigma\left(\hat{i}_t-E_t\{\hat{\pi}_{t+1}\}\right)+E_t\{\hat{y}_{t+1}\}
%	\end{align*}
\begin{itemize}
	\item Iterating forward, the current output gap depends negatively on the gap between the real interest rate and the natural rate of interest (assuming return to steady state):
	\begin{align*}
		\hat{y}_t=-\sigma E_t\left\{\sum_{s=0}^{\infty}\left(\hat{r}_{t+s+1}\right)\right\}
	\end{align*}
	\item If you want to figure out what happens to output in the NK model, you need to figure out what happens to the path of the real interest rate.
	\begin{itemize}
		\item Output gap determined purely by intertemporal substitution. Not old Keynesian marginal propensities to consume / invest.
		\item Intuition also works well for larger NK models.
	\end{itemize}
\end{itemize}
\end{frame}

%%%%%%%%%%%%%%%%%%%%%%%%%%%%%%%%%%%%%%%%%%%%%%%%%%
\section{Three Equation Model}
%%%%%%%%%%%%%%%%%%%%%%%%%%%%%%%%%%%%%%%%%%%%%%%%%%

\begin{frame}
\frametitle{The Three Equation Model}
\begin{itemize}
	\item In sum, the log-linearized NK model boils down to three equations:
	\begin{align*}
		\hat{y}_t &=-\sigma[\hat{i}_t-E_t\{\hat{\pi}_{t+1}\}]+E_t\{\hat{y}_{t+1}\} \\
		\hat{\pi}_t&=\kappa (\hat{y}_t-\hat{y}_t^{flex}) +\beta E_t \{\hat{\pi}_{t+1}\} \\
		\hat{i}_t&=\phi_\pi\hat{\pi}_t+v_t \\
	\end{align*}
	with three unknowns: $\hat{i}_t$, $\hat{y}_t$, and $\hat{\pi}_t$  and an exogenous driving
process for the output gap $\hat{y}_t^{flex}$ ($=\frac{1+\varphi}{\gamma+\varphi}\hat{a}_t$) and the monetary policy shock $\hat{v}_t$.
	\item Key new ingredient is NK Phillips curve:
	\begin{itemize}
		\item $\beta E_t \{\hat{\pi}_{t+1}\}$: Price setters forward looking.
		\item $\kappa\hat{y}_t$: Output  $\uparrow\;\Rightarrow$ MC $\uparrow\;\Rightarrow$ markups $\downarrow\;\Rightarrow$ raise prices
	\end{itemize}
	\item Determinacy: similar condition to lecture 1. See Gali.
	\item Note: Gali writes everything in terms of ``gaps'', $\tilde{y}_t = \hat{y}_t-\hat{y}_t^{flex}$.
\end{itemize}
\end{frame}

\begin{frame}
\frametitle{Special Case: $\kappa\rightarrow\infty$}
\begin{itemize}
%	\item To understand how the NK model works let us focus first on special cases. 
	\item Equivalent to flexible prices: $\lambda=0$.
	\item The NK Phillips Curve becomes:
	\begin{align*}
		\hat{y}_t = \hat{y}_t^{flex}&=\frac{1+\varphi}{\gamma+\varphi}\hat{a}_t 
	\end{align*}
	\item Output fluctuations arise only from productivity fluctuations.
	\item Monetary variables $v_t$ have no real effect:
	\begin{itemize}
		\item Drop in $v_t$ lowers nominal rate and real interest rate.
		\item All else equal increases output and marginal cost.
		\item Prices today rise with marginal cost so that $\pi_{t+1}$ is low and the real interest rate is constant.
		\item[$\Rightarrow$] With constant real interest rate output is unchanged.
	\end{itemize}
\end{itemize}
\end{frame}

\begin{frame}
\frametitle{Special Case: $\kappa=0$}
\begin{itemize}
	\item Equivalent to perfectly rigid prices: $\theta=1$.  The NK Phillips Curve becomes $\hat{\pi}_t=0$.
	\item Now output is demand determined:
	\begin{align*}
		\hat{y}_t &=-\sigma\hat{i}_t +E_t\{\hat{y}_{t+1}\} \\
		\hat{i}_t&=v_t 
	\end{align*}
	\item If $v_t = \rho_v v_{t-1}+\epsilon_t$, then
	\begin{align*}
		\hat{y}_t = -\frac{\sigma}{1-\rho_v}v_t
	\end{align*}
	\item Monetary variables $\hat{v}_t$ have a real effect:
	\begin{itemize}
		\item Drop in $v_t$ lowers nominal rate and real interest rate.
		\item Inflation is constant so output expands with the lower real interest rate.
	\end{itemize}
	\item With rigid prices, output is independent of productivity fluctuations.
\end{itemize}
\end{frame}

\begin{frame}
\frametitle{Intermediate $\kappa$}
\begin{itemize}
	\item Assume
	\begin{align*}
		v_t = \rho_v v_{t-1}+\epsilon_t \text{ and }\hat{a}_{t}=0
	\end{align*}
	\item Guess reduced form policy functions:
	\begin{align*}
		\hat{y}_t=\psi_{yv}v_t\text{ and }\hat{\pi}_t=\psi_{\pi v}v_t
	\end{align*}
	\item This gives:
	\begin{align*}
		\psi_{\pi v}&=\kappa\psi_{yv}+\beta\rho_v\psi_{\pi v}\\
		\psi_{y v}&=-\sigma(\phi_\pi \psi_{\pi v} + 1 -\rho_v \psi_{\pi v})+\rho_v \psi_{y v}
	\end{align*}
	\item Solving by method of undetermined coeffs:
	\begin{align*}
		\psi_{y v}=-(1-\beta\rho_v)\Psi_v\text{ and }\psi_{\pi v}=-\kappa \Psi_v
	\end{align*}
	\begin{align*}
		\text{where } \Psi_v=\frac{1}{(1-\beta\rho_v)\gamma(1-\rho_v)+\kappa(\phi_\pi-\rho_v)}>0
	\end{align*}
\end{itemize}
\end{frame}


\begin{frame}
\frametitle{Epistemiology}
\begin{itemize}
	\item NK model a response to the Lucas critique: ``fully'' optimizing agents but monetary policy has real effect.
	\item Extensive debate on how well NK model fits the data.
	\begin{itemize}
		\item Centers on much more complex ``medium-scale'' models.
		\item These are the simple NK model at its core with many additional ``bells and whistles'' (capital, habits, indexation, rigid wages, government, etc).  See references in the syllabus.
		\item Everyone agrees the simple three equation model does not match the data well.
%		\item See references in the syllabus.
	\end{itemize}
	\item Should view simple model as an organizing framework.
	\begin{itemize}
		\item Communicate results: everyone knows this model and how it works.
		\item How should policy respond to shocks? Why?
		\item Interpret policy actions through lens of NK model.
	\end{itemize}
\end{itemize}
\end{frame}
\end{document}






